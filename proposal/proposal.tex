\documentclass[12pt,a4paper,twoside]{article}
\usepackage[pdfborder={0 0 0}]{hyperref}
\usepackage[margin=25mm]{geometry}
\usepackage{graphicx}
\usepackage{parskip}
\usepackage[UKenglish]{isodate}
\usepackage{listings}

\cleanlookdateon
\lstset{basicstyle=\ttfamily, breaklines=true}

\begin{document}

\begin{center}
\Large
Computer Science Tripos -- Part II -- Project Proposal\\[4mm]
\LARGE
Implementing a dependently typed programming language\\[4mm]

\large
J.~Hinshelwood, Selwyn College

Originator:

\today
\end{center}

\vspace{5mm}

\textbf{Project Supervisor:} Dr N.~Krishnaswami

\textbf{Director of Studies:} Dr R.~Watts

\textbf{Project Overseers:} Prof. G.~Winskel \& Dr R.~Mortier

% Main document

\section*{Introduction}

Dependent types are an extension to the type systems present in many programming languages.
Systems with parametric polymorphism allow these types to be indexed by other types (e.g. \lstinline{List Bool}), meaning functions can be defined generically over an infinite range of types.
Dependent types extend this idea further, allowing types to be indexed over values, as well as types.
One of the most common examples of a dependent type is the type of a List with a known length - (e.g. \lstinline{Vect 4 Bool}).

There are many advantages of adding dependent types to a programming language.
Often performance gains can be made when types are more expressive, as the compiler knows more about them and can make better optimisations.
For example, if length-indexed vectors are used, the compiler does not need to insert index out-of-bounds checks, since it can statically prove they will not occur.
More expressive type systems also allow the compiler to reject a greater range of erroneous programs.
If a \lstinline{Matrix} type indexed by a width and height is used, the compiler can disallow matrix multiplication of matrices with improper dimensions.

My project is to design and implement a language which uses a dependent type system.
I will write a type checker and interpreter for the language in OCaml.
To demonstrate, I will write some examples which showcase the type system's capabilities.

\section*{Starting point}

Various dependently typed languages already exist.
There are also a few papers on the topic, which I may refer to.
I will be writing my implementation in OCaml.
I do not anticipate needing to use any additional libraries at this point.

\section*{Resources required}

For this project I will primarily use my own laptop to develop the implementation and to write the dissertation.
I will use git to version control my work and make regular backups to a GitHub repository.
I require no other special resources.

\section*{Work to be done}

The project breaks down into the following sub-projects:

\begin{enumerate}

\item Formalise the semantics of a dependently typed language, based on the dependently typed \(\lambda\)-calculus.

\item Implement a type checker for the language.

\item Implement an interpreter or compiler for the language.

\item Write examples demonstrating the usefulness of dependent types.

\end{enumerate}

\section*{Success criteria}

The project will be a success if I have formalised the semantics of a dependently typed language, as well as having implemented a type checker and interpreter for the language.


\section*{Possible extensions}

If I achieve my success criteria early, I will try and add the following extensions to the language.

\begin{itemize}
	\item Datatype declarations
	\item Pattern matching
	\item Type inference
	\item Implicit arguments
\end{itemize}

If some of these features get implemented, I will also spend more time writing and updating examples in the language.
These will get easier to read as the language develops more features.


\section*{Timetable}

The planned starting date is 22/10/2018.

\begin{enumerate}

\item \textbf{Michaelmas weeks 2--4} Read papers.

\item \textbf{Michaelmas weeks 5--6} Define formal semantics.

\item \textbf{Michaelmas weeks 7--8} Start implementation of type checker.

\item \textbf{Michaelmas vacation} Finish type checker and implement interpreter.

\item \textbf{Lent weeks 0--2} Finish interpreter and write progress report.

\item \textbf{Lent weeks 3--5} Write examples in the language.

\item \textbf{Lent weeks 6--8} Begin extensions.

\item \textbf{Easter vacation} Finish extensions and write dissertation main chapters.

\item \textbf{Easter term 0--2}  Complete dissertation.

\item \textbf{Easter term 3} Proof reading and submission.

\end{enumerate}

\end{document}
